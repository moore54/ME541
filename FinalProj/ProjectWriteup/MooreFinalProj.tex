
\documentclass[10pt,english]{article}
\usepackage[bottom=0.8in,margin=0.8in]{geometry}
\geometry{a4paper}
\usepackage[english]{babel}
%\usepackage{indentfirst}
\selectlanguage{english}
\usepackage{multicol}
%\usepackage{varioref}% smart page, figure, table, and equation referencing
\usepackage[colorlinks]{hyperref}% hyperlinks [must be loaded after dropping]
\hypersetup{colorlinks,breaklinks,
citecolor = blue,
urlcolor=blue,
linkcolor=blue}
\usepackage[noabbrev]{cleveref}
\usepackage{float}

\usepackage{placeins}
\usepackage[utf8]{inputenc}

%Julia input
\usepackage[T1]{fontenc}
\usepackage{beramono}
\usepackage{listings}
\usepackage[usenames,dvipsnames]{xcolor}

\definecolor{myred}{RGB}{200,50,50}
\definecolor{mycomments}{RGB}{75,100,175}

%%
%% Julia definition (c) 2014 Jubobs
%%
\lstdefinelanguage{Julia}%
{morekeywords={abstract,break,case,catch,const,continue,do,else,elseif,%
		end,export,false,for,function,immutable,import,importall,if,in,%
		macro,module,otherwise,quote,return,switch,true,try,type,typealias,%
		using,while},%
	sensitive=true,%
	alsoother={\$},%
	morecomment=[l]\#,%
	morecomment=[n]{\#=}{=\#},%
	morestring=[s]{"}{"},%
	morestring=[m]{'}{'},%
}[keywords,comments,strings]%

\lstset{%
	language       		 	  = Julia,
	basicstyle           	  = \ttfamily,
	tabsize				    	 =2,
	breaklines			 	 =true,
	keywordstyle     	 = \bfseries\color{myred},
	stringstyle      	 	  = \color{magenta},
	commentstyle    	= \color{mycomments},
	showstringspaces = false,
}
% End Julia input

\usepackage{graphicx}

\usepackage{amssymb}
\usepackage{authblk}
\graphicspath{{../figures/}} % allows figures to be placed in a different folder


\title{\vspace{-20pt}2D SIMPLE CFD Program Verification with Poiseuille Flow}
\author{Kevin R. Moore}
\affil{\vspace{-10pt}Brigham Young University}
\renewcommand\Authands{, }
\date{\today}

\begin{document}

\maketitle
\vspace{-30pt}

\begin{table*}[h]
\vspace{20pt}
\centering
  \begin{tabular}{lcl}
    \textbf{Name} & \textbf{Value} & \textbf{Description}  \\
    $\alpha_{U}$ & 0.6 & U-relaxation factor  \\
    $\alpha_{V}$ & 0.6 & V-relaxation factor  \\
    $\alpha_{P}$ & 1.0 & P-relaxation factor  \\
    Convergence Criteria  & 1E-6 & $l^2$ Norm of P' \\
    Model Accuracy Criteria  & \% & Outlet U-Velocity Integral \% Error  \\
    Finest Mesh  & 4096 & Nodes (64x64) \\

  \end{tabular}
  \caption{Summary of solver parameters.}
  \label{tab:params}
\end{table*}

\vspace{-20pt}

\section{Contour Plots}

\begin{figure}[H]
\centering
\includegraphics[trim={.0cm 0.5cm .0cm 0cm},clip,width=0.9\textwidth]{p_contour4}
\vspace{-5pt}
\caption{Pressure contour of finest solution with 4096 nodes.}
\label{f:3b}
\end{figure}

\vspace{-10pt}
\begin{figure}[H]
\centering
\includegraphics[trim={.0cm 1.0cm .0cm 0cm},clip,width=0.9\textwidth]{u_contour4}
\vspace{-5pt}
\caption{U-Velocity contour of finest solution.}
\label{f:3b}
\end{figure}

\vspace{-10pt}
\begin{figure}[H]
\centering
\includegraphics[trim={.0cm 1.0cm .0cm 0cm},clip,width=0.9\textwidth]{v_contour4}
\vspace{-5pt}
\caption{V-Velocity contour of finest solution.}
\label{f:3b}
\end{figure}

\vspace{-10pt}
\begin{figure}[H]
\centering
\includegraphics[trim={.0cm 1.0cm .0cm 0cm},clip,width=0.9\textwidth]{stream4}
\vspace{-5pt}
\caption{Stream plot showing U and V velocities' relative direction and pressure in the color.}
\label{f:3b}
\end{figure}




\FloatBarrier
\section{Convergence Criteria Independence}



\noindent Norm of pressure correction was used as the main convergence criteria calculated as:
    \begin{equation}
      Convergence\,Criteria = \sqrt{\sum_{n=1}^N \bf P_{\rm n}^{'2}}
    \end{equation}


\noindent Analytical pressure and U-Velocity profile was calculated by:
\begin{equation}
            L = Out-In
\end{equation}
\begin{equation}
            D_H = 2 (Top-Bot)
            \end{equation}
            \begin{equation}
            Re_D = \rho U_0 D_H / \mu
            \end{equation}
            \begin{equation}
            f = 24/Re_D
            \end{equation}
            \begin{equation}
            \frac{\Delta p}{\Delta x} = \frac{4 f L (0.5 \rho U_0^2)}{L D_H}
            \end{equation}
            \begin{equation}
            u_{analytical} = \frac{{\Delta p}/{\Delta x}}{(2 \mu) y (Top-y)}
            \end{equation}

\begin{figure}[H]
\centering
\includegraphics[trim={.0cm 0.0cm .0cm 0cm},clip,width=0.75\textwidth]{outlet_4}
\vspace{-5pt}
\caption{Analytical versus numerical solution; solution progression from left to right towards convergence.  Updated every 10 iterations. }
\label{f:3b}
\end{figure}

\noindent Integral error in percentage defined as:

    \begin{equation}
       Integral\,Error\,(\%) = \frac{\int_{Bot}^{Top}{(u_{analytical}-u_{numerical})}}{\int_{Bot}^{Top}{u_{analytical}}} \times 100
    \end{equation}

\noindent Numerical values were splined using cubic b-splines to keep error calculations consistent over 100 samples regardless of grid size.  U-Velocity at outlet was used.


\begin{figure}[H]
\centering
\includegraphics[trim={.0cm .5cm .0cm 0cm},clip,width=0.85\textwidth]{error_4}
\vspace{-5pt}
\caption{Percent integral error of numerical and analytical, converges with about 0.077\% error for the finest grid, which is shown here. Updated every 10 iterations. (Recall convergence criteria was $l^2$ norm of P' was less than 1E-6 though based on the trend shown here either a tighter or difference criteria may lead to even better results)}
\label{f:3b}
\end{figure}

\FloatBarrier
\vspace{5pt}
\section{Grid Independence}


\begin{figure}[H]
\centering
\includegraphics[trim={.0cm 0cm .0cm 0cm},clip,width=0.75\textwidth]{GridCon_Error}
\vspace{-5pt}
\caption{Doubling number of nodes in each direction cuts error by a factor of approximately 4 for each iteration.}
\label{f:3b}
\end{figure}


\begin{figure}[H]
\centering
\includegraphics[trim={.0cm 0cm .0cm 0cm},clip,width=0.75\textwidth]{GridCon_Iters}
\vspace{-5pt}
\caption{Doubling number of nodes in each direction increases required iterations for convergence in a linear manner.}
\label{f:3b}
\end{figure}

\section{Validation}

\subsection{Pressure}
According to the analytical equation above  $\Delta P = 0.006\,pa$.  I calculate, on the numerical side, $\Delta P = 0.00645\,pa$, which I calculated from the mean pressure on the inlet minus the mean pressure on the outlet.  Using only the centerline pressures, the numerical $\Delta P = 0.00587\,pa$.\\


\subsection{Outlet U-Velocity}
\noindent Analytical U-Velocity

\noindent [0.0, 4.65088e-5, 0.000137329, 0.00022522, 0.000310181, 0.000392212, 0.000471313, 0.000547485, 0.000620728, 0.00069104, 0.000758423, 0.000822876, 0.000884399, 0.000942993, 0.000998657, 0.00105139, 0.0011012, 0.00114807, 0.00119202, 0.00123303, 0.00127112, 0.00130627, 0.0013385, 0.0013678, 0.00139417, 0.0014176, 0.00143811, 0.00145569, 0.00147034, 0.00148206, 0.00149084, 0.0014967, 0.00149963, 0.00149963, 0.0014967, 0.00149084, 0.00148206, 0.00147034, 0.00145569, 0.00143811, 0.0014176, 0.00139417, 0.0013678, 0.0013385, 0.00130627, 0.00127112, 0.00123303, 0.00119202, 0.00114807, 0.0011012, 0.00105139, 0.000998657, 0.000942993, 0.000884399, 0.000822876, 0.000758423, 0.00069104, 0.000620728, 0.000547485, 0.000471313, 0.000392212, 0.000310181, 0.00022522, 0.000137329, 4.65088e-5, 0.0]\\

\newpage
\noindent Numerical U-Velocity

\noindent [0.0, 4.69612e-5, 0.000137936, 0.000225965, 0.000311048, 0.000393186, 0.000472377, 0.000548623, 0.000621922, 0.000692275, 0.00075968, 0.000824139, 0.000885651, 0.000944217, 0.000999837, 0.00105251, 0.00110224, 0.00114903, 0.00119288, 0.00123378, 0.00127175, 0.00130678, 0.00133887, 0.00136803, 0.00139426, 0.00141756, 0.00143793, 0.00145538, 0.00146989, 0.00148149, 0.00149016, 0.00149591, 0.00149874, 0.00149866, 0.00149565, 0.00148972, 0.00148088, 0.00146912, 0.00145444, 0.00143684, 0.00141633, 0.00139289, 0.00136654, 0.00133726, 0.00130506, 0.00126995, 0.0012319, 0.00119094, 0.00114705, 0.00110024, 0.0010505, 0.00099783, 0.000942236, 0.000883715, 0.000822266, 0.000757889, 0.000690583, 0.00062035, 0.00054719, 0.000471103, 0.00039209, 0.000310152, 0.000225291, 0.000137509, 4.6808e-5, 0.0]


\section{Code}
\noindent \bf{Code can be found at: https://github.com/moore54/ME541.git under /FinalProj/Solve.jl as well as in Appendix A.  Code will be removed from github when grades are posted to prevent future unauthorized use.}


\section{Appendix A. Included Code}
\lstinputlisting[language=Julia]{../Solve.jl}

\end{document}
